\documentclass{www2010-submission}

\usepackage{graphicx}
% \usepackage[left=2cm,top=1cm,right=3cm]{geometry}

\begin{document}
\setlength{\parindent}{0pt}
\setlength{\parskip}{.5ex plus 0.5ex minus 0.2ex}


% \setlength{\parskip}{8pt}
% \setlength{\parsep}{8pt}


\title{ drawnTogether -- a collaborative approach to virtual graffiti }

\author{ Jeremy Kelley, Kyumin Lee, \& Xiheng Zhang }

\date{\today}

\maketitle

\section{ User Study 1}
\subsection{Interviewees}
We conducted nine people:
A. male, an iphone 3GS user
B. male, a computer science graduate student
C. female, a person who likes drawing
D. male, an iPhone 3GS user
E. male, a computer science graduate student
F. male, a computer science graduate student
G. (Jeremy add three people or you can refrase this section)
H.
J.

\subsection{conducting user study}
(Xiheng suggests we each play a role in the test process)
greeter: introduce what the system does, and what the goals are for this study to participants in my group.
facilitator:
computer: move the three interface around inside the board.
observer:
 
We interviewed nine people during three days. In each day, we interviewed three people. 
First day, we met three people (A,B and C), introduced our application and then asked them weather they find the work is interesting or not.

A and C felt it's interesting. But B think though it is a funny application, there must be enough situations that the system can be used. i.e, more scenarios needed. He provide a scenario. Image a group of architecture students enter into one building to make a sketch of the inside structure of this building. Each student can find whether the viewer around him or her has been drawn or not. If not, he can add his sketch on the screen of the application. If yes, he may ignore it and go to other places to draw or make some correction based on others' drawings. Though GPS may not work inside building, it is still a nice scenario if the drawings take place outside.

Then we gave the paper mock-up device and four pens with different colors (red,  yellow, blue, green) to each person individually. With the drawing screen shown in the simulated device initially, just let the users to figure out how to draw in this mock-up application.
A asked me, with the initial screen show up, can I draw now? What's the status of my "pen"?
B said nothing, but immediately touched the setting button. Then the "computer" moved the "setting" interface on the screen of the mock-up device. C just randomly pick up a pen (blue) and draw on the device. Our facilitator told her it's not allowed to choose the color in the drawing page, users should go to setting page to choose color.

In the setting page, all of them have a clear view of choosing color and erase. But C asked about tag. Our facilitator explain to her what it is. B asked about how to type letters into the tag text area. ``Computer'' said there will be a pop-up keyboard to let you input tag, just like all other iphone applications. A asked when should he tag? facilitator told him before beginning to draw.

For the filter part, our facilitator explained what it is used for. B asked where these tags come from. How many of them would display in the screen? Do the server contain all the tags users made? In the map, A asked how to show the sketches in different locations in one map? Pop-up window? He suggested the size should be the same as the drawing page. 

Next day, we interviewed three people (D,E and F) in HRBB. We introduced the purpose of our application. Before we explained it in detail, we asked them whether they can understand how to use it. They were able to figure it out in most parts. But, one of interviewees said ``figure others'' in the setting page is ambiguous and recommended to show ``filter existing tags'' on the page. In addition, he suggested it is better for the application to provide filtering sketches based on ranks or time (last year or this year). For example, users can vote whether each sketch is good or bad. A sketch with ``good'' many times from users will be ranked high. 

After we got their oppinions, we explained our application in detail. One of interviewes strongly recommended moving color palette in the setting page to the bottom of drawing page because he and other users will often change the color while they are drawling. If its location is not changed, he has to click or touch three times (go to setting page, select a color and go back to drawing papge). Even though there is trade-off providing smaller drawing part, he said it is better. He also said map page is supposed to be the first showned page after he logged in the application because he wanted to know how many people already drawned sketches in a certain location. He said the number will affect whether he will draw skech or not. Actually, he prefered adding sketches to create a new sketch. 

In the end of the interview, most of them said they are very interested in the application and are willing to buy it, if it is sold.

%%%%%%%%%%%%%%%%%%%
% \bibliographystyle{plain}
% \bibliography{../bibli} 
\end{document}
