\documentclass{article}

\usepackage{graphicx}
\usepackage[left=2cm,top=1cm,right=3cm]{geometry}

\begin{document}
\setlength{\parindent}{0pt}
\setlength{\parskip}{.5ex plus 0.5ex minus 0.2ex}


% \setlength{\parskip}{8pt}
% \setlength{\parsep}{8pt}


\title{ Revised Proposal }

\author{ Jeremy Kelley, Kyumin Lee, \& Xiheng Zhang }

\date{\today}

\maketitle

\section{Revised Proposal}

We are going to develop a collaborative artistic application for the iPhone 3GS. This application allows users raw sketches on their iPhone and upload the sketch as well as their global position and orientation at that moment, to a server. The sketch can be anything the user wants to draw. It can be the view he/she is seeing, or the idea he/she is thinking about, or his mood and his feeling. The purpose of this application is to help users collaborate and communicate better with others in the same network, or let actors be more aware of what happened around based on context information and more involved in social activities. The context information contains both temporal and spatial information. Each sketch will be timestamped at the point it is uploaded to the server.

The spatial data has:
\begin{enumerate}
\item the current position the user locates, indicated by x (latitude), y (longitude) and z (altitude) available via GPS in the device.
\item the current orientation the user faces available via the magnetic compass within the device.
\end{enumerate}
The orientation is important because view would be affected given different orientation.

Just like common social networks, users have his/her own networks in our system. i.e., one may only interested in his/her friends' sketches or sketches of people who have a same interest with him/her. Also, users can choose his/her sketch to be public, private or only share with friends. Due to the huge amount of data in server, users need the ability to filter contributions based on his/her preferences. This refine process will be discussed in our next step of work.

The specific activities users will conduct using our system include:
\begin{enumerate}
\item Hold iphone in hand, stay at any location GPS can track.
\item On the iphone screen, make some sketch or create a brief text annotation.
\item For sketch, users can paint in a limited number of colors and then tag the entire sketch.
\item After the sketch is done, users can upload it.
\item Users can retrieve entries based on a simple query interface. Thus conditions involve: social relations, tag, time, and present location.
\end{enumerate}

Here is an initial description of the components we will need to prototype for the system (hardware and software)
\begin{enumerate}
\item Client interface
	\begin{itemize}
	\item iPhone 3GS
	\item  Sketch application
	\end{itemize}

\item Web service
	\begin{itemize}
	\item Operating system: Linux
	\item Web server: Lighttpd
	\item Web application: custom service build using Django framework
	\item Database: MySQL
	\item Code management tool during development: git
	\end{itemize}
\end{enumerate}

% %%%%%%%%%%%%%%%%%%%
\bibliographystyle{plain}
\bibliography{../bibli} 
\end{document}
