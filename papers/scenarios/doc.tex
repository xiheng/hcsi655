\documentclass{article}

\usepackage{graphicx}
\usepackage[left=2cm,top=1cm,right=3cm]{geometry}

\begin{document}
\setlength{\parindent}{0pt}
\setlength{\parskip}{.5ex plus 0.5ex minus 0.2ex}


% \setlength{\parskip}{8pt}
% \setlength{\parsep}{8pt}


\title{ Scenarios }

\author{ Jeremy Kelley, Kyumin Lee \& Xiheng Zhang }

\date{\today}

\maketitle

\section{Scenarios}
We can apply our application to various scenarios. In this section, we show two of them.   

\subsection{Scenario 1: collaborative drawing}

Alice and Bob are students at Texas A\&M University and are friends. Alice wanted to draw the MSC building before it is remodeled, so she went near to the building and sketched it using our application. She did not finish the drawing because of a class. While she is going to a classroom, she sends a SMS message to Bob. The message is ``Could you finish the sketch for the MSC building in location X?'' Fortunately, he is passing in front of the building and starts to add some sketch into the figure, which Alice drew. Sometimes later, he finishes to draw the building and sends a SMS message to her. The message is ``I'm done!'' After the class is finished, she goes to the location in which she drew the building. She can view what they drew and is satisfied about what they did.

\subsection{scenario 2: recalling a old building}

Now the MSC building is under construction. George is a new student at TAMU. He hears about the old MSC building from senior friends, but he does not know what exactly it looked like and what happened near to the MSC building in the past. His senior friend tells him to use our application to view the scenes because some students drew the building and scenes near the building in the past, leaving some comments. He can view stored scenes and filter them by a drawer, time and so on. Even though the building is under construction, new students including George can recall it.


\end{document}
