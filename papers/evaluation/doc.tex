\documentclass{article}

\usepackage{graphicx}
\usepackage[left=2cm,top=1cm,right=3cm]{geometry}

\begin{document}
\setlength{\parindent}{0pt}
\setlength{\parskip}{.5ex plus 0.5ex minus 0.2ex}


% \setlength{\parskip}{8pt}
% \setlength{\parsep}{8pt}


\title{ Evaluation Plan }

\author{ Jeremy Kelley, Kyumin Lee, \& Xiheng Zhang }

\date{\today}

\maketitle

\section{ Evaluation }

\subsection{Overview}

All people who own an iphone may be interested in this system, especially young people in university. If interviews and tests need to be conducted during the course of the project, we prefer do a small group of field study. We first introduce our system and make them aware how useful and funny our system is.  After users have much experience with the application, we will get their feedback about this system based on interviews. We may input some random sketches into our server repository before users are able to view submitted drawings.

\subsection{Hypotheses}
The sketch application on iphone 3GS should be manipulated conveniently and effectively. On the screen of iphone 3GS, uses can make sketch freely with four colors. The most significant part of this application is that collaboration among  users should happen in real time, even with large number of people drawing for the same object at the same time.

\subsection{Paper Mockup Evaluation}
Before we write the application, we will conduct an interview with 5-10 people for the mock interactive interface on paper. We will observe what they will draw on the paper. Of course, the available area on the paper is the same size as the screen of iphone 3GS. This part will be done before Nov. 24.

\subsection{Field Study}
After the first version of our application is done (after Dec.1 and before Dec. 5), we will hands on 5-10 people who have iphone 3GS and interested in our newly developed application to test our real application. Ask them to install it. Basically, we will obverse how people are using it and ask them their opinions about our system.

\subsubsection{Observe users}
We will help users to install the application, but when the installation is finished, we will not teach them how to use the application, just observe how users will manipulate in their iphone 3GS screen in each step. During this process, of course, users will ask questions about what would happen when touching the interactive screen or in order to show what the user want the screen to be, what kind of operation should be done? Based on those questions, we will know what part should be revised in the application. Also, we will see if the sketch will be updated automatically in real-time among other users' iphone screen.

\subsubsection{Interview}

After users finish playing with our application, we will ask some questions as
the part of interview.  Here is an initial list of structured-interview
questions:
\begin{itemize}
\item Is this idea interesting? Are you having fun with the new iphone 3GS application?
\item Are the sketch setting options enough for you to draw? If not, what else should be considered?
\item How you like the user interface? Any improvement needed?
\item If you are the designer, what other affordances do you like to add into our application?
\item Will you recommend this application to your friends and play with them together?
\item Is there any problem for this application? Any suggested solutions?
\end{itemize}
 
% %%%%%%%%%%%%%%%%%%%
\bibliographystyle{plain}
\bibliography{../bibli} 
\end{document}
