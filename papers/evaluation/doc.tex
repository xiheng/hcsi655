\documentclass{article}

\usepackage{graphicx}
\usepackage[left=2cm,top=1cm,right=3cm]{geometry}

\begin{document}
\setlength{\parindent}{0pt}
\setlength{\parskip}{.5ex plus 0.5ex minus 0.2ex}


% \setlength{\parskip}{8pt}
% \setlength{\parsep}{8pt}


\title{ Evaluation Plan }

\author{ Jeremy Kelley, Kyumin Lee, \& Xiheng Zhang }

\date{\today}

\maketitle

\section{Overview}

All people who own an iphone may be interested in this system, especially young people in university. If some interviews and tests need to be conducted during the course of the project, we will make interviews or tests at those places with large stream of people, e.g. library. We first introduce our system and make them aware how useful and funny our system is.  After that, we will get their feedback about this system based on interviews. We may need to input some random submissions into our server before users are able to view submitted content.

\section{Hypotheses}
The sketch application on iphone 3GS should be manipulated conveniently and effectively. On the screen of iphone 3GS, uses can make sketch freely with four colors. The most significant part of this application is that collaboration among  users should happen in real time, even with large number of people drawing for the same object at the same time.

\section{Field Study}
We will find students who have iphone 3GS in front of library or at the lobby of Zachry Building. Ask people who are interested in our newly developed application to install it. Basically, we will obverse how people are using it and ask them their opinions about our system.

\subsection{Observe users}
We will help users to install the application, but when the installation is finished, we will not teach them how to use the application. We just observe how users will manipulate their iphone 3GS screen in each step. During this process, of course, users will ask questions about how to touch the interactive screen or what kind of operation can come up with a result of what the users want it to be. Since it is a collaborative system, we need to observe one than more users at the same time. We will see if the sketch will update automatically in real-time among other users' iphone screen.

\subsection{Interview}

After users finish playing with our application, we will ask some questions as the part of interview.

Here is a list of structured-interview questions:

Is this idea interesting? Are you having fun with the new iphone 3GS application?

Are the sketch setting options enough for you to draw? If not, what else should be considered?

How you like the user interface? Any improvement needed?

If you are the designer, what other affordances do you like to add into our application?

Will you recommend this application to your friends and play with them together?

\section {Analytical evaluation}

We will upload the new application to the internet and only ask other iphone developers to use it. Besides, we will ask classmates who is also in CSCE655 to evaluate our application. There will be some feedback from their reviews. We will also ask them for suggested solutions of certain problems. Further more, we will make a questionnaire dealing with quantitative and qualitative data for our application and collect useful data from their feedback.
 
% %%%%%%%%%%%%%%%%%%%
\bibliographystyle{plain}
\bibliography{../bibli} 
\end{document}
