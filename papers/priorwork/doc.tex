\documentclass{article}

\usepackage{graphicx}
\usepackage[left=2cm,top=1cm,right=3cm]{geometry}

\begin{document}
\setlength{\parindent}{0pt}
\setlength{\parskip}{.5ex plus 0.5ex minus 0.2ex}


% \setlength{\parskip}{8pt}
% \setlength{\parsep}{8pt}


\title{ Prior Work }

\author{ Jeremy Kelley, Kyumin Lee \& Xiheng Zhang }

\date{\today}

\maketitle

\section{Prior Work}

Zurita et al. \cite{sketching:zurita} presents MCSketcher, a mobile collaborative sketching system, using handheld devices in an ad-hoc network. People can draw sketches with collaborators in the same page in the system. They can take a picture and use it in the background of the drawing. The system supports gesture recognition for basic navigation gestures, and selecting and resizing items. Collaborative work is only available in an ad-hoc network. In other words, friends or collaborator can only work together in close distance. Our application connects to a server to store what users draw and to retrieve what they drew in the past using cellular networks. One can collaborate with not only friends, but also any people.

Kim and Dey \cite{augmented_reality:kim} presents displaying augmented reality based car navigation information on windshield. Augmented reality helps elder people can concentrate to drive a car and easily follow directions from a navigation system. Augmented reality is one of ways to display various information in a view. Likewise, overlapping sketches can help for users to know what has happened in a certain location, even though augmented reality is not exactly applied to our application. In addition, our application provides users with some filters such as name of a drawer, time and so on, so that they can select what they want to view.

Bast\'{e}a-Forte and Yen \cite{brainstorming:marcello} presents a collaborative sketching tool. Each user has a Tablet PC to draw some sketches and to view shared sketches. Each user�s drawling is synchronized, so that users can view the same drawling. The collaborative sketching tool helps each user�s contribution is equalized although it reduces total number of sketches. Our application enables users to collaborate drawing no matter what their relationship is. 

% %%%%%%%%%%%%%%%%%%%
\bibliographystyle{plain}
\bibliography{../bibli} 
\end{document}
