\documentclass{www2010-submission}

\usepackage{graphicx}
% \usepackage[left=2cm,top=1cm,right=3cm]{geometry}

\begin{document}
\setlength{\parindent}{0pt}
\setlength{\parskip}{.5ex plus 0.5ex minus 0.2ex}


% \setlength{\parskip}{8pt}
% \setlength{\parsep}{8pt}


\title{ drawnTogether -- a collaborative approach to virtual graffiti }

\author{ Jeremy Kelley, Kyumin Lee, \& Xiheng Zhang }

\date{\today}

\maketitle

\section{ Ethnographic data gathering }
We interviewed three people using a semi-structured nature. One of them came from China and is a student at Texas A\&M University. Our questions and his answers are as follows: ``Have you used a drawing tool before? If yes, which tool have you used and how often have you used it?'' He said he usually uses Paint application and sometimes uses Photoshop.  Overall, he uses the drawing tools one a month. ``Why do you like to use Paint more often than Photoshop?'' He said Paint application is simple and easy to use, but Photoshop has too many functions and is sophisticated to use. Additionally, after he heard about our application, he said it is helpful to prove a function to search a sketch drawn by friends without going to a certain location. One is a Chinese girl who has never played iphone before and her major is not computer science, but economics. But she is very interested in drawing and painting. We asked her questions as following. Q1. Have you drawn with any devices besides paper before? If yes, what are they? A1. Yes. I drew on a small toy board which has an eraser when I was a little child. Randomly, I use windows paint application. Most often, I use professional software, such as spss to draw professional diagram. Q2. Which kind of way do you prefer to draw? A2. Of course, more like a pen and a paper drawing, better experience with sketch. I don't like to use mouse to draw stroke at all. If I can have a pen to draw on the computer screen, I would definitely draw a good picture. Q3. Have you ever thought about collaborating with others to draw on the same screen? A3. That's interesting. If there is an application allow many people drawing together, personally, I wish the system could tell my strokes apart from others'. Q4. Image you are standing outside, holding a drawing device in your hand, what will you draw on the screen? A4. I don't quite understand your question. Why are you asking this? I will draw anything I like. Q5. I mean will you draw what you see or just make some random scratches? A5. I guess I will draw what I see in the first place. Even if not what I see, I would like to draw something related to what I see. Q6. If there is a such application, do you care about what others draw? Whose drawings do you care about most? A6. I do care about others' sketch, that's one aspect of collaboration, right? First, I care about my friends' drawing. Then people who share the same interests with me. Then I talked about the basic idea of our project. She said she would care more about what people around her were drawing.

Analysis of findings

\section{Storyboards}
Jeremy, we have to refine the this section. I just copy and paste our previous scenarios.
We can apply our application to various scenarios. In this section, we show two
of them.

\subsection{Scenario 1: collaborative drawing}

Alice and Bob are students at Texas A\&M University and are friends. Alice
wanted to draw the MSC building before it is remodeled, so she went near to the
building and sketched it using our application. She did not finish the drawing
because of a class. While she is going to a classroom, she sends a SMS message
to Bob. The message is ``Could you finish the sketch for the MSC building in
location X?'' Fortunately, he is passing in front of the building and starts to
add some sketch into the figure, which Alice drew. Sometimes later, he finishes
to draw the building and sends a SMS message to her. The message is ``I'm
done!'' After the class is finished, she goes to the location in which she drew
the building. She can view what they drew and is satisfied about what they did.

\subsection{scenario 2: recalling a old building}

Now the MSC building is under construction. George is a new student at TAMU. He
hears about the old MSC building from senior friends, but he does not know what
exactly it looked like and what happened near to the MSC building in the past.
His senior friend tells him to use our application to view the scenes because
some students drew the building and scenes near the building in the past,
leaving some comments. He can view stored scenes and filter them by a drawer,
time and so on. Even though the building is under construction, new students
including George can recall it.



%%%%%%%%%%%%%%%%%%%
\bibliographystyle{plain}
\bibliography{../bibli} 
\end{document}
