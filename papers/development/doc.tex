\documentclass{article}

\usepackage{graphicx}
\usepackage[left=2cm,top=1cm,right=3cm]{geometry}

\begin{document}
\setlength{\parindent}{0pt}
\setlength{\parskip}{.5ex plus 0.5ex minus 0.2ex}


% \setlength{\parskip}{8pt}
% \setlength{\parsep}{8pt}


\title{ Development Plan }

\author{ Jeremy Kelley, Kyumin Lee, \& Xiheng Zhang }

\date{\today}

\maketitle

\section{Development Plan}

\subsection{ Application Environment and Development Tools }

\subsubsection{ Mobile Client }

Since the prototype is intended for the iPhone OS, there are few options for
development environments.  The path of least resistance (and greatest
productivity) is for the application to be written in Objective C, and built
using Apple's iPhone SDK toolchain integrated into the Xcode IDE.  It is
possible to develop iPhone applications outside of Xcode, using other text
editors and foregoing the builtin debugging integration that Xcode offers, but
to compile and link, one is still required to use the provided tools by Apple.
For this reason, our team will focus on the Xcode environment and the full
stack that it has to offer.  In particular, we will be using the following:

\begin{itemize}
\item development workstations will run OSX
\item Xcode
\item iPhone 3G S
\item iPhone Simulator
\end{itemize}

\subsubsection{ Web Service }

In order for the applications on the client devices to share content, a
centralized web service is required.  There are many options within this realm,
however, for ease of development since time is short on this prototype, this
web service will be built using the following:

\begin{itemize}
\item Python scripting language
\item Django Web Framework
\item Lighttpd HTTP Server
\item MySQL Database
\item JSON for data transport
\end{itemize}

This stack of tools has been widely utilized in numerous installations and in
particular, the Python language has proven itself to be quite powerful, yet
easy to learn.

\subsubsection{ Version Control \& Source Code Management }

In order to facilitate sharing our work, we have created a repository at
GitHub\footnote{http://github.com}.  This allows us to use Git, a distributed
version control system, for all of our created content.  Our source code,
scripts, and even our papers (written in \LaTeX) will be stored in this shared
repository.  This allows us, as a distributed team working in disparate
locations, to share all content and stay up-to-date with others contributions.

Git also provides issue tracking that we have begun using in an effort to track
progress and discussion on multiple concurrent items.

\subsection{ Development Milestones  }

\begin{itemize}
\item 17 Nov -- Storyboards and Lo-fi Prototypes
\item 19 Nov -- Webservice completed and API locked for data transfer
\item 23 Nov -- pre-Alpha ``Sketchable'' interface working on Sim with fabricated geo data
\item 24 Nov -- Initial User Feedback incorporated into final designs
\item 26 Nov -- Early ``Alpha'' version of the mobile client ready
\item 1 Dec -- Prototype due

\end{itemize}

\subsection{ Application Architecture }

The heart of the mobile application will be the integration of a live video
feed overlaid by user contributed sketches for the current location of the
device.  The application will rely on Apple's CoreLocation service to provide
the location and orientation information in order to retrieve user content from
the web service.  We will run an NSTimer that will create
NSInvocationOperations at regular intervals to submit any user created content
in the background during viewing and to retrieve any new content for the user's
current geographic area.

We intend to incorporate a simple user identity component using Twitter.  This
will allow us to attach a user to a sketch without having to manage the
identity infrastructure.  By having identity information attached to each
contribution, we will also have the ability to allow users to filter based upon
the user.

There will be a tagging component, although it will be limited in this initial
prototype.  The user will be able to create a set of tags that will be applied
to sketches created after tag assignment.

Finally, a map view will be provided which will indicate to the user sketches
near to them, to allow for ``sketch tours'' or other artistic discoveries.


%%%%%%%%%%%%%%%%%%%
\bibliographystyle{plain}
\bibliography{../bibli} 
\end{document}
